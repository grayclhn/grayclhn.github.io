\begin{filecontents}{syllabus-2014.bib}
@book{CB02,
  title =	 {Statistical Inference},
  author =	 {Casella, George and Berger, Roger L},
  year =	 2002,
  publisher =	 {Duxbury Press},
  edition =	 {2nd}
}

@Article{Fi26,
  author =	 {R.A. Fisher},
  title =	 {The Arrangement of Field Experiments},
  journal =	 {Journal of the Ministry of Agriculture of Great
                  Britain},
  year =	 1926,
  volume =	 33,
  pages =	 {503-513},
  note =	 {Available at
                  \url{http://digital.library.adelaide.edu.au/dspace/handle/2440/15191}}
}

@Book{Fr09,
  author =	 {David A. Freedman},
  title =	 {Statistical Models: Theory and Practice},
  publisher =	 {Cambridge University Press},
  year =	 2009,
  edition =	 {revised}
}

@article{Fr91,
  author =	 {Freedman, David A.},
  title =	 {Statistical Models and Shoe Leather},
  journal =	 {Sociological Methodology},
  volume =	 21,
  pages =	 {291-313},
  note =	 {Available at
                  \url{http://www.jstor.org/stable/270939}},
  year =	 1991,
}

@book{GH07,
  title =	 {Data Analysis Using Regression and
                  Multilevel/Hierarchical Models},
  author =	 {Gelman, Andrew and Hill, Jennifer},
  year =	 2007,
  publisher =	 {Cambridge University Press}
}

@Book{Ga97,
  author =	 {A. Ronald Gallant},
  title =	 {An Introduction to Econometric Theory},
  publisher =	 {Princeton University Press},
  year =	 1997
}

@Book{Gr12,
  author =	 {William H. Greene},
  title =	 {Econometric Analysis},
  publisher =	 {Prentice Hall},
  year =	 2012,
  edition =	 {7th}
}

@Article{IW09,
  author =	 {Guido W. Imbens and Jeffrey M. Wooldridge},
  title =	 {Recent Developments in the Econometrics of Program
                  Evaluation},
  journal =	 {Journal of Economic Literature},
  year =	 2009,
  volume =	 47,
  number =	 1,
  pages =	 {5-86},
  month =	 {March},
  note =	 {Available at
                  \url{http://ideas.repec.org/a/aea/jeclit/v47y2009i1p5-86.html}}
}

@book{Ro09,
  title =	 {Design of Observational Studies},
  author =	 {Rosenbaum, Paul R.},
  year =	 2009,
  publisher =	 {Springer}
}

@article{Ru05,
  title =	 {Causal inference using potential outcomes},
  author =	 {Rubin, Donald B.},
  journal =	 {Journal of the American Statistical Association},
  volume =	 100,
  number =	 469,
  year =	 2005,
  note =	 {Available at
                  \url{http://www.jstor.org/stable/27590541}}
}

@Book{eflp-core,
  author =	 {Gray Calhoun},
  title =	 {Core Econometrics},
  publisher =	 {Econometrics Free Library Project,
                  \url{http://www.econometricslibrary.org}},
  year =	 2014,
  note =	 {Version 0.7.2}
}
\end{filecontents}

\documentclass[nofonts,nols]{tufte-handout}

\makeatletter
\newlength{\fullwidthlength}
\AtBeginDocument{\setlength{\fullwidthlength}{\@tufte@fullwidth}}
\makeatother

\title[Econ 671 Syllabus, Fall 2014]{Economics 671, PhD Econometrics I}
\date{Fall 2014. Course homepage: \protect\homepage}

\morefloats
\usepackage{
  booktabs,
  enumitem,
  fancyvrb,
  fancyhdr,
  marginfix,
  tabularx,
  url,
  verbatim,
  xcolor,
}
\usepackage[T1]{fontenc}
\usepackage[utf8x]{inputenc}
\usepackage[bitstream-charter]{mathdesign}
\usepackage[letterspace=35]{microtype}
\usepackage[osf]{sourcecodepro}
\usepackage{sourcesanspro}

% Change to ragged-right without inline math breaks; this helps avoid
% awkward line breaks or extensions into the right margin in math or
% code.
\RaggedRight
\relpenalty=10000
\binoppenalty=10000
\setlength{\parindent}{1em}

\renewcommand{\smallcaps}[1]{\allcaps{#1}}
\renewcommand{\allcaps}[1]{\textls{\MakeUppercase{#1}}}

\renewcommand{\FancyVerbFormatLine}[1]{\hspace{\leftmargini}#1}
\fvset{frame=leftline,rulecolor=\color{lightgray}}

\urlstyle{same}
\DeclareUrlCommand\url{%
\def\UrlLeft{\guillemotleft}%
\def\UrlRight{\guillemotright}%
\urlstyle{same}}

\DisableLigatures{family = tt*}
\frenchspacing

\newcommand{\homepage}{\url{http://www.econ.iastate.edu/~gcalhoun/671}}
\newcommand{\OH}{\allcaps{OH}}
\newcommand{\RAT}{\allcaps{RAT}}
\newcommand{\TA}{\allcaps{TA}}
\newcommand{\TBD}{\allcaps{TBD}}
\newcommand{\TBL}{\allcaps{TBL}}

\renewcommand{\labelitemi}{{$\circ$}}
\renewcommand{\labelitemii}{\footnotesize$\circ$}
\renewcommand{\labelitemiii}{\textperiodcentered}
\renewcommand{\labelitemiv}{\footnotesize\textperiodcentered}

\begin{document}
\noindent%
\begin{minipage}{\fullwidthlength}
\maketitle
\end{minipage}

\begin{table}[h]
\begin{tabularx}{\textwidth}{rXX}
  \toprule
         & Gray Calhoun         & Yang He (\TA)       \\
  \midrule
  email  & gcalhoun@iastate.edu & yanghe@iastate.edu  \\
  phone  & (515) 294-6271       &                     \\
  office & 467 Heady            & 271 Heady           \\
  \OH    & Tu 2--3:15           & \TBD                \\
  \bottomrule
\end{tabularx}
\caption{Instructor and \TA\ contact information.}
\end{table}

\noindent%
Welcome to Econ 671! This class has three goals. You are going to
study and learn fundamental techniques in econometrics and statistics
so that you can use them in your future research.  You are also going
to learn some of the basic theoretical concepts in econometrics so
that you can understand new techniques when you encounter them in
future classes and later in your career.  And, finally, you're going
to learn how to use a computer to do statistical and econometric
analysis.

If you have questions about the course material, the best times to
address them are in the scheduled class meetings or during office
hours. We can probably resolve questions or concerns about the course
administration over email, but if you have urgent questions please
call me or stop by my office.

\section{Textbooks and software}

This class has two required textbooks, \citep{CB02} and \citep{Gr12},
and two recommended textbooks, \citep{Ga97} and \citep{Fr09}.%
\footnote{There is a bibliography with full citations at the end of the
  syllabus.} %
These are all available at the university bookstore but can be
purchased cheaper online. There are a few other required readings that
are available through the course homepage. Additionally,
\citep{eflp-core} is a disorganized collection of notes from previous
years that I've taught this class and is available for free download
online at \url{http://www.econometricslibrary.org}. These notes are
neither required nor necessarily recommended, but you may find them
helpful.  The course \emph{Reading guide} has a short overview of the
different readings.

You are also going to start to learn computer programming in this
class. The \TA\ will teach R, a specialized language that's designed
for statistical analysis, in some of the Friday discussion sessions.%
\footnote{R can be downloaded for free from
  \url{http://www.r-project.org}.} %
You may also want to look at Julia and Python,%
\footnote{Julia is available for free at \url{http://julialang.org}.}%
\footnote{For Python, you'll especially want the packages at
  \url{http://www.scipy.org}.} %
they are also free software and are also designed for statistical and
scientific computing, but have different strengths than R.

\section{Grading}

This course uses the \emph{Team-Based Learning} (\TBL) instructional
strategy, which is probably different from instruction styles you've
had before. Most of the content is covered individually with readings
and short problems completed outside of class.  Most of the activities
and projects, which would conventionally be done as out-of-class
homework and group projects, are done in teams during class.

There will be six short multiple-choice \emph{Readiness-Assurance
  Tests} (\RAT s) at the \textbf{beginning} of each unit of material;
these will be taken as individuals first, then as a team. There will
also be six main team projects at the end of each unit, and an
individual midterm and final exam. For team tests and projects, all
members of the team will receive the same score.

Scores in three areas will determine the grades: \emph{Individual
  Performance}, \emph{Team Performance} and constructive behavior as
determined by \emph{Peer Evaluations}.

\subsection{Setting Grade Weights}

Representatives from each team will set the percentage of the course
grade that will be determined by scores in each of the major
performance areas during the first class period. Team representatives
will also decide on the relative weight of the Readiness Assurance
Tests and the exams within the Individual Performance area.

Grade weights will be set for the class using the following
procedures:
\begin{enumerate}
\item Each team will set preliminary weights and select a member to
  meet with other teams' representatives.
\item Team representatives will meet in the center of the room and
  develop a consensus (i.e., every representative has to be in
  agreement) about the grade weights for the class as a whole.
\item The only limitations on your grade weight decisions are listed
  in the table:
  \begin{enumerate}
  \item A minimum of 20\% of the total grade must be assigned to each
    major performance area.
  \item Within the individual performance area, at least 20\% of the
    grade must be based on each exam and on the total of the
    individual \RAT s.
  \end{enumerate}
\end{enumerate}
Table~\ref{weights} summarizes these rules and provides space to enter
the weights that you all agree on.

\begin{table}[t]
\newcommand{\rl}{\rule{.75cm}{.4pt}}
\begin{tabularx}{\textwidth}{Xccc}
\toprule
Component                       & Weight & Min. (\%) & Max (\%) \\
\midrule
                                                                \\
\textbf{Individual performance} & \rl    & 20        & 60       \\
\cmidrule(r){1-1}
Individual \RAT s               & \rl    & 20        & 60       \\
Midterm exam                    & \rl    & 20        & 60       \\
Final exam                      & \rl    & 20        & 60       \\
                                                                \\
\textbf{Team performance}       & \rl    & 20        & 60       \\
\cmidrule(r){1-1}
Team \RAT s                     & 30\%                          \\
Team projects                   & 70\%                          \\
                                                                \\
\textbf{Peer evaluations}       & \rl    & 20        & 60       \\
\bottomrule                                                     \\
\end{tabularx}
\caption{Weights for each component of the course grades --- the
  specific weights will be determined by the class as described in
  the syllabus. The three components of ``individual performance''
  must add up to 100\%, with a minimum weight of 20\% on each exam
  and on the overall individual \RAT\ scores. The weights of the three
  main performance areas (in bold) must also add up to 100\% and must
  each be at least 20\%.}
\label{weights}
\end{table}

\subsection{Peer evaluations}
Each individual will rate the contributions all of the other members
of their teams during the final exam.  Individual Peer Evaluation
scores will be the average of the points they receive from the members
of their team.  Assuming arbitrarily that: 1) constructive behavior is
worth 10 points, and 2) that there six members in a team, an example
of this procedure would be as follows:

Each individual must assign a total of 50 points to the other five
members in their team.  Raters must differentiate some scores in their
ratings (This means that each rater would have to give at least one
score of 11 or higher --- with a maximum of 15 --- and at least one
score of 9 or lower --- with a minimum of 5). The Team Maintenance
scores will produce differences in grades only within teams.  As a
result, team-members can't help everyone in their team get an A by
giving them a high peer evaluation scores.  The only way for everyone
in a team to earn an A is by doing an outstanding job on the
individual exams and team exams and projects.

\subsection{Determination of final grades}

The final grades will be determined as follows:
\begin{enumerate}
\item A raw total score will be computed for each student in each
  major performance area. (In the Individual Performance area, this
  will be a weighted combination of the sum of the individual
  Readiness Assurance Test scores and the final exam score, in the
  Team Performance area, this will be the sum of the scores on each of
  the graded team assignments and the Team Maintenance score will be
  the average of the peer evaluations received from the other members
  of his or her team.)
\item Students' total scores will be computed by multiplying the raw
  scores in each area by the grade ``weight'' set by the class (see
  above).
\item Course grades will be based on each student's standing in the
  overall distribution of total individual scores within the class.
  The actual impact of any score on an individual student's final
  grade depends on both his or her actual score and also how high or
  low he or she scores relative to other members of the class.  The
  conventional practice of 90\% is an A, 80\% is a B, etc. simply does
  not apply.
\end{enumerate}

\section{Planned schedule}

The course is broken up into six parts with required reading for each
one, as listed in Table~\ref{tab1}. Additional ungraded prep work may
be assigned as the semester goes along.

\begin{table*}[t]
  \begin{tabularx}{\textwidth}{lrX}
    \toprule
    Topic                                         & \RAT  & Required reading                                                                                          \\
    \midrule
    Probability                                   & 9/02  & \citep{Gr12} B, D; \citep{CB02} 5.1--5.3, 5.5                                                             \\
    Statistical estimation                        & 9/23  & \citep{CB02} 7, 10.1, 10.2                                                                                \\
    Statistical inference                         & 10/07 & \citep{CB02} 8, 9, 10.3, 10.4                                                                             \\
    Linear regression                             & 10/21 & \citep{Gr12} 2--4, 5.1--5.7, 9.1, 9.2                                                                     \\
    Regression modeling                           & 11/11 & \citep{Gr12} 5.8--5.11, 6, 9.1--9.3, 10.1--10.3; \citep{GH07} 6                                           \\
    Program evaluation                            & 12/02 & \citep{Gr12} 8.1--8.3; \citep{Fi26}; \citep{Fr91}; \citep{IW09} 1--4, 5.1--5.3, 5.11--5.13; \citep{Ru05}; \\
                                                  &       & \citep{Ro09} 1, 19, Summary                                                                               \\
                                                                                                                                                                      \\
    Exam                                          & Date  & Exam time                                                                                                 \\
    \midrule
    \multicolumn{2}{l}{Midterm \hfill Fri. 10/24} & 9:00a -- 11:00                                                                                                    \\
    \multicolumn{2}{l}{Final \hfill Mon. 12/15}   & 7:30a -- 9:30 (sucks, I know)                                                                                     \\
   \bottomrule
  \end{tabularx}
  \caption{List of major units and required reading for the class.
    There is a bibliography with full citations at the end of the syllabus.
    The individual chapters and articles are available on on the course
    homepage (or on Google).}
  \label{tab1}
\end{table*}

The Friday meetings will usually cover programming and software
development. They will often be unrelated to the main class --- think
of them as a parallel class --- but we will sometimes use those
sessions for review or to continue a team project. A tentative plan
for the sessions is listed in Table~\ref{tab2}.

\begin{table}[t]
  \caption{Tentative list of topics covered during Friday review
    session. Classes that are \emph{emphasized} will be held in the
    regular classroom; the rest will be held in the computer lab.}
  \label{tab2}
  \begin{tabular}{lrlr}
    \toprule
    First half of semester          & Week & Second half                & Week      \\
    \midrule
    Intro to computing resources    & 1    & Intro to the Unix shell    & 10        \\
    Programming in R                & 2    & Version control with Git   & 11        \\
    Graphics in R                   & 3    & More Git                   & 12        \\
    Intro to data management        & 4    & \emph{Class meeting}       & \emph{13} \\
    More data management            & 5    & \TBD\ --- discuss with \TA & 14        \\
    LaTeX                           & 6    & \emph{Final exam review}   & \emph{15} \\
    Reproducible research \& Knitr  & 7                                             \\
    \emph{Midterm review}           & \emph{8}                                      \\
    \emph{Midterm}                  & \emph{9}                                      \\
    \bottomrule
  \end{tabular}
\end{table}

\section{License and copyright}

To the extent possible under law, Gray Calhoun, the author, has waived
all copyright and related or neighboring rights to this
document. Anyone is free to reuse some or all of this syllabus to
teach a similar class, or for any other purpose. You can download the
LaTeX source code for this file from the course homepage, \homepage.

\newpage

\section{University policies}

The following policies apply to \emph{every} course at Iowa State
University. They are listed here for your convenience and reference.

\subsection{Academic dishonesty}

The class will follow Iowa State University's policy on academic
dishonesty.  Anyone suspected of academic dishonesty will be reported
to the Dean of Students Office,
\url{http://www.dso.iastate.edu/ja/academic/misconduct.html}.

\subsection{Disability accommodation}

This material can be provided to you in alternative format. Anyone who
anticipates difficulties with the content or format of the course due
to a physical or learning disability should see me immediately in
order to work out a plan. You may also want to contact the Disability
Resources (\allcaps{DR}) office, located on the main floor of the
Student Services Building, Room 1076 or call them at 515-294-7220.

\subsection{Dead week}

For academic programs, the last week of classes is considered to be a
normal week in the semester except that in developing their syllabi
faculty shall consider the following guidelines:

\begin{itemize}
\item Mandatory final examinations in any course may not be given
  during Dead Week except for laboratory courses and for those classes
  meeting once a week only and for which there is no contact during
  the normal final exam week. Take home final exams and small quizzes
  are generally acceptable. (For example, quizzes worth no more than
  10 percent of the final grade and/or that cover no more than
  one-fourth of assigned reading material in the course could be
  given.)
\item Major course assignments should be assigned prior to Dead Week
  (major assignments include major research papers, projects,
  etc.). Any modifications to assignments should be made in a timely
  fashion to give students adequate time to complete the assignments.
\item Major course assignments should be due no later than the Friday
  prior to Dead Week. Exceptions include class presentations by
  students, semester-long projects such as a design project in lieu of
  a final, and extensions of the deadline requested by students.
\end{itemize}

\subsection{Harassment and discrimination}

Iowa State University strives to maintain our campus as a place of
work and study for faculty, staff, and students that is free of all
forms of prohibited discrimination and harassment based upon race,
ethnicity, sex (including sexual assault), pregnancy, color, religion,
national origin, physical or mental disability, age, marital status,
sexual orientation, gender identity, genetic information, or status as
a U.S. veteran. Any student who has concerns about such behavior
should contact his/her instructor, Student Assistance at 515-294-1020,
or the Office of Equal Opportunity and Compliance at 515-294-7612.

\subsection{Religious accommodation}

If an academic or work requirement conflicts with your religious
practices and/or observances, you may request reasonable
accommodations. Your request must be in writing, and your instructor
or supervisor will review the request.  You or your instructor may
also seek assistance from the Dean of Students Office or the Office of
Equal Opportunity and Compliance.

\subsection{Contact information}

If you feel that any of your rights as a student have been violated,
please email \url{academicissues@iastate.edu}.

\newpage

\bibliographystyle{alpha}
\bibliography{syllabus-2014}

\end{document}
