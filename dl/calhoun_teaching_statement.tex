\documentclass[12pt]{article}%
\author{Gray Calhoun}
\title{Statement of Teaching Philosophy}
\frenchspacing
\usepackage[T1]{fontenc}
\usepackage{textcomp} % to get the right copyright, etc.
\usepackage{lucidabr}
%\renewcommand{\familydefault}{\sfdefault}
\DeclareEncodingSubset{TS1}{hlh}{1}

% Semi-detailed preamble follows
\usepackage{
    calc,
    fancyhdr,
    hyperref,
    ragged2e,
    url,
}
\usepackage[margin=1in]{geometry}
\usepackage[letterspace=30]{microtype}

\pagestyle{fancy}
\cfoot{\oldstylenums{\thepage}}
\renewcommand{\headrulewidth}{0pt}

\RaggedRight
\urlstyle{same} %<- Do not monospace urls.
\newcommand{\allcaps}[1]{\textls{\MakeUppercase{#1}}}
\newcommand{\smallcaps}[1]{\textsc{\MakeLowercase{#1}}}

% Set spacing and size of sections and subsections. (Don't use
% anything below subsection.)
\usepackage[tiny,raggedright]{titlesec}
\titleformat{\section}{\bf\lsstyle\uppercase}{}{}{}{}
\titleformat{\subsection}{\bf}{}{}{}{}
\titlespacing*{\section}{0pt}{2\baselineskip}{\baselineskip}
\titlespacing*{\subsection}{0pt}{\baselineskip}{\itemsep}
\setlength\parindent{1em}

% Change the displayed title to emphasize the author and radically
% reduce spacing.
\makeatletter
\def\maketitle{%
\begin{center}%
\par{\textls{\MakeUppercase{\textbf{\@title}}}}%
\vspace{\itemsep}%
\par{\@author, \@date}%
\end{center}%
}
\makeatother
% semi-detailed preamble over

\begin{document}
\maketitle

\noindent%
My teaching philosophy has been heavily influenced by the ``Team-Based
Learning'' instructional style (TBL), which I learned about from a
series of workshops run at Iowa State in 2014. TBL is a ``flipped
classroom'' approach; students are expected to familiarize themselves
with course material outside of class and deepen their understanding
by applying that material in the classroom. But TBL differs from other
flipped classroom teaching styles in the way that it structures
classroom exercises and students' individual study sessions and, I
believe, these features make it especially well suited for my
instructional goals and for many aspects of university
education. Since taking these workshops, I have converted my required
PhD econometrics course to TBL (Econ 671); I am in the process of
converting my Principles of Macroeconomics course (Econ 102); and I
plan to convert my graduate econometrics elective (Econ 674) in the
future. I have also continued to learn more about TBL, frequently
attend meetings of the TBL ``Learning Community'' organized on campus,
and have presented some of my course material to the TBL Learning
Community for feedback (on October 3, 2014, October 30, 2015, and
December 9, 2016).

In the first part of this section, I will explain my teaching
principles and goals. The second part of this section will briefly
describe TBL and explain how it is effective in meeting those
goals. And the third part discusses areas of teaching that occur
outside of formal coursework---another area of teaching that I have
been heavily involved in. In addition to the courses I have taught at
Iowa State, I have been the advisor for two PhD students, both of whom
have graduated and taken research positions in academia, and am
currently advising another. I have also served on 15 graduate student
committees in the Economics and Statistics departments (including
current students); and I have organized several informal reading
groups and workshops for graduate students.

\section*{Teaching principles and goals}

I try to follow three general principles when I teach. First, I
believe it is important to give students the opportunity to solve
challenging, open-ended assignments that match real world use of the
material covered in the course. These assignments serve several
purposes: they help students integrate the different components of the
course and understand them better; and they directly help students
understand how the material can be used in practice, giving them a
better understanding of economics in general and making it more likely
that they will recognize situations where they can apply the skills
they have learned in the class. Second, I try to structure the course
and its assignments to motivate students and to prepare them to be
successful in solving challenging and realistic open-ended
assignments. Motivating students that are very comfortable with the
material and motivating students that are struggling with the material
typically require different approaches, but both are important. Third,
I design the course so that the students and the instructor get
frequent and accurate feedback about the students' performance and
level of understanding.

Obviously, a challenging, open-ended, and realistic assignment for
Principles of Macroeconomics is very different than one for a
PhD-level econometrics course, but I have found that emphasizing these
principles helps both sets of students: the Principles of
Macroeconomics students learn skills that they can use later in
college and after graduation, and the PhD students are shown how they
can apply the course material to their own research.

\section*{TBL description and examples from my current teaching}

A TBL class has a few distinctive features. (More details are
available at the main TBL website,
\url{http://www.teambasedlearning.org}, along with references to
additional resources.)

\begin{itemize}
\item The semester is split into 4-7 major sections, each lasting 2--4
  weeks. For PhD Econometrics, for example, I use the following 6
  sections:
  \begin{enumerate}
  \item Introduction to Probability Theory (3 weeks)
  \item Sampling and asymptotics (2 weeks)
  \item Statistical inference and estimation (3 weeks)
  \item Finite-sample properties of linear regression (2 weeks)
  \item Asymptotic properties of linear regression (3 weeks)
  \item Causal inference using linear regression (2 weeks)
  \end{enumerate}

\item Students are assigned reading (and potentially homework) before
  each section, then spend most of the class meetings working with
  their team on challenging group assignments.

\item Each of the 6 major sections starts with a structured quiz and
  review session that lasts one or two class meetings. This review
  ensures that students are familiar with and prepared to use the
  material in the team assignments. After that review, the remaining
  class periods are spent on team assignments, where students will
  learn how to use this material effectively.

\item The team assignments follow several principles to make them as
  useful as possible; assignments that are easily split into parts
  that can be delegated to individual members of a team are
  discouraged in TBL, as are those that require teams to make a final
  product that is particularly complicated, time-consuming, or
  difficult to evaluate. (Group papers and group presentations are
  strongly discouraged in TBL for those reasons.) For most
  assignments, teams are instead asked to make a small number of
  decisions based on some amount of background and context-specific
  information. For example, one of the assignments towards the end of
  my Principles of Macroeconomics class has students assume the role
  of an advisor to the Icelandic government during the recent
  financial crisis. It culminates in the question:

  \emph{Which of the following policies would you advise the Icelandic
    government would be best for the economy in the middle of a
    financial crisis? (There is going to be a recession for sure, this
    question is all about managing its damage.)
    \begin{enumerate}
    \item Allow the Krona/Euro exchange rate to be determined by the
      market, and use monetary and fiscal policy to offset the
      recession.
    \item Plan to sell Krona to keep the exchange rate from moving too
      quickly; then use monetary policy to offset the recession.
    \item Plan to buy Krona to keep the exchange rate from moving too
      quickly; then use monetary policy to offset the recession.
    \item Peg the exchange rate at a new level using monetary policy;
      use fiscal policy to offset the recession.
    \item Decrease government spending to restore faith in financial
      markets.
    \end{enumerate}}

  To effectively answer this question, students need to apply specific
  and detailed knowledge from the course and to evaluate and discuss
  each possibility. But to report its answer, a team just needs to
  choose the appropriate letter. This allows the students to focus
  entirely on understanding and using the course material, not on
  coordinating the effort to report their conclusions.

  However, this is not a pure ``multiple choice'' question. Teams will
  be expected to explain their answers to the class and to evaluate
  the other teams' choices as part of the exercise and to get credit
  for their answer. Requiring teams to report a single choice is done
  to facilitate discussion within the teams and within the class
  overall.

\item Team construction follows several principles as well. Students
  are grouped into relatively large teams (5--7 students each) at the
  beginning of the course and keep those teams for the entire
  semester. The teams are assembled transparently by the instructor to
  have a diverse mix of backgrounds and skills in each team. Since
  they work together for the entire semester, students have enough
  time to build strong relationships with their teammates, leading to
  higher performance through more effective teamwork and greater
  individual accountability.
\end{itemize}

I have found that this structure is very effective at meeting my
instructional goals and principles. These teams of students are able
to solve very difficult real-world questions that very few of the
students would be able to solve individually. Each student is able to
contribute meaningfully to the team, and the interaction between the
students helps them all understand the material more
deeply. Similarly, when teams are working together effectively they
can accomplish more in a single lecture than students working
individually, so the assignment can incorporate more of the
meta-skills that will be necessary for students to apply the material
later in their careers---framing the problem so that it can be solved
using some of the models they've learned earlier in the semester, for
example---that are difficult to fit into class otherwise. Moreover,
when students are working on assignments, the TAs and I circulate
among the teams to answer questions, discuss the approach they are
taking, and (occasionally) give them additional problems if they think
that they've finished the current assignment. This, along with the
full-class discussions and the review sessions in the review process,
provides an enormous amount of feedback to the students and the
instructors on their level of understanding and performance, making it
easier for the students to adjust their effort levels and allowing me
to provide additional background knowledge if it is necessary.

Furthermore, the emphasis on team assignments has served as a very
powerful motivator for students of every level of individual
performance. This motivation can be increased even more when the
assignment can be structured so that teams are competing against each
other directly. (Another one of TBL's recommendations.) Direct
competition is easy to incorporate into many assignments in economics
and econometrics; in one assignment for Econ 671, for example, teams
construct and estimate a linear regression model for forecasting
state-level unemployment, and are partly graded on the forecast's
accuracy; and in another assignment teams use probabilistic and
statistical models to bid for risky assets in an auction. Both of
these assignments have had very high levels of engagement, enthusiasm,
and effort from the students.

\section*{Mentoring and advising graduate students}

I have been very active in working with graduate students outside of
class as well. In addition to advising several PhD students and
serving on graduate student committees (in both the Economics and the
Statistics departments), I have organized reading groups for graduate
students and faculty that have targeted specific areas of
econometrics, one on the bootstrap during the spring 2016 semester and
one that met later that summer on machine learning. After speaking
with faculty at other universities about their graduate programs, I
have also started to organize a workshop for our PhD students that
will start meeting in the 2016 Fall semester. This workshop will be
used for PhD students to present their research regularly to the
econometrics faculty and to review and discuss relevant papers that
are presented in our departmental seminar, and faculty at other
institutions have found that similar workshops have greatly benefited
their graduate programs. This workshop is being organized in
consultation with several other faculty members in the department, but
under my own initiative. We expect it to lead to much better research
by our graduate students and a higher level of polish and confidence
on the job market.

The same principles that I try to follow in the classroom apply in
these less formal settings as well: I try to give students a safe
environment for them to practice and perform the skills they need to
develop professionally and I try to pay attention to the students'
motivation and energy level and give them frequent and accurate
feedback.

\end{document}
